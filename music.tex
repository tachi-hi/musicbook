\documentclass[10pt]{jbook}
%\usepackage[top=2cm,bottom=2cm,left=2.5cm,right=2.5cm]{geometry}

%\usepackage{type1ec} %Type1フォント(英数字や記号などの1バイト文字を表示するフォント形式)を使用するとき
%\usepackage[OT2,T1]{fontenc}	% OT2はロシア語(キリル文字)を使用する場合に指定する
%\usepackage{textcomp}	% TS1エンコーディング(T1エンコーディング以外の各種記号)を使用する場合
%\usepackage{hiraprop}	% Times系フォントを使用する場合
%\usepackage[english,french,russian]{babel}

\title{\LaTeX + Lilypond (備忘録)}
\date{}
\author{}


\begin{document}
\maketitle
\chapter{\LaTeX の文書中で楽譜を表示するサンプルです.}
%\selectlanguage{russian}
%\foreignlanguage{russian}{\CYRA}

減3、短3、長3、増3
\begin{lilypond}[quote,fragment,staffsize=16] 
\relative c'{
<cis es>1 <cis e> <des f> <des fis>
}
\end{lilypond}


\begin{lilypond}[quote,fragment,staffsize=16, notime] % "notime" option removes time signature and bar
\relative c'{
c1^\markup{一度} \glissando c1 \bar "||"
c ^\markup{増一度} \glissando cis \bar "||"
c?^\markup{長二度} \glissando d \bar "||"
c ^\markup{増二度} \glissando dis \bar "||"
c ^\markup{長三度} \glissando e \bar "||"
c ^\markup{完全四度} \glissando f \bar "||"
c ^\markup{増四度} \glissando fis \bar "||"
c ^\markup{完全五度} \glissando g' \bar "||"
c,^\markup{増五度} \glissando gis' \bar "||"
c,^\markup{長六度} \glissando a' \bar "||"
c,^\markup{増六度} \glissando ais' \bar "||"
c,^\markup{長七度} \glissando b' \bar "||"
c,^\markup{完全八度} \glissando c' \bar "||"
}
\end{lilypond}

和声的短音階
\begin{lilypond}[quote,fragment,staffsize=16] 
\relative c''{
\clef treble
\key gis \minor
gis4 ais b cis dis e fisis gis
}
\end{lilypond}

モーツァルト
\begin{lilypond}[quote,fragment,staffsize=16] 
\relative c''{
\clef treble
\time 2/4
\key g \minor
r4 es8 d d4 es8 d d4 es8 d d4 bes' r bes8 a g4 g8 f es4 es8 d
}
\end{lilypond}

\section{和声}

SATB

\begin{lilypond}[quote,fragment,staffsize=16] 
\new StaffGroup <<
	\new Staff<<
		\clef soprano
		\set Staff.instrumentName = \markup{ Soprano }
		\relative c' {
			c'2 c b c
		}
	>>
	\new Staff<<
		\clef alto
		\set Staff.instrumentName = \markup{ Alto }
		\relative c' {
			e2 f d e
		}
	>>
	\new Staff <<
		\clef tenor
		\set Staff.instrumentName = \markup{ Tenor }
		\relative c{
			g'2 a g g
		}
	>>
	\new Staff<<
		\clef bass
		\set Staff.instrumentName = \markup{ Bass }
		\relative c{
			c2 f, g c
		}
	>>
>>
\end{lilypond}

大譜表

\begin{lilypond}[quote,fragment,staffsize=16] 
\new PianoStaff <<
	\new Staff<<
		\clef treble
		\relative c' {
			c'2 c b c
		}\\
		\relative c' {
			e2 f d e
		}
	>>
	\new Staff <<
		\clef bass
		\relative c{
			g'2 a g g
		}\\
		\relative c{
			c2 f, g c
		}
	>>
>>
\end{lilypond}

\[
a + b + c = d
\]
\end{document}


